\section{Вывод}
\label{sec:conclusion}

В результате изучения такой сферы как предсказания временных рядов были рассмотрены основные характеристики временных рядов, которые используются при их прогнозировании. Можно сказать, что задача прогнозирования, как и любая другая задача, возникающая в процессе работы с данными — во многом творческая и уж точно исследовательская. Несмотря на обилие формальных метрик качества и способов оценки параметров, для каждого временного ряда часто приходится подбирать и пробовать что-то своё.

На данный момент классические методы предсказания могут быть заменены более современными методами машинного обучения. Однако это не означает, что описанные выше методы и подходы не подходят для предсказания временных рядов. Данные подходы позволяют обеспечивать достаточно точный результат прогнозирования временных рядов, учитывающий сезонности, циклы и тренды, максимально уменьшая влияние ошибки на предсказание.

Предсказания временных рядов позволяет делать достаточно точные прогнозы, которые способны удовлетворять требованиям бизнеса, такие как предсказания трендов в экономике, для прогноза сколько серверов понадобится online-сервису через год, каков будет спрос на каждый товар в гипермаркете, или для постановки целей и оценки работы команды.

Для прогноза временных рядов могут использоваться как ручные расчёты, так и заранее подготовленные библиотеки в программирование, например библиотека statsmodels в языке Python. Данный модуль предоставляет широкий набор средств и методов для проведения статистического анализа и эконометрики. В целом для небольших исследований пакет statsmodels может быть использован и получить, с достаточной долей точности.

Таким образом - использование предсказания временных рядов используя всё описанное выше достаточно легко может производиться современными средствами программирования и решать множество задач, где необходимо предсказать какие-либо события, где имеются заранее подготовленные замеры с фиксированной длиной промежутков замеров.

