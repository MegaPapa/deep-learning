\section{Стационарность временного ряда}
\label{sec:stationary}

Очень важным свойством временного ряда, помимо автокорреляции, является его стационарность.

Ряд $y_1, y_2 ... y_T$ стационарен, если любое распределение $y_t, ..., y_{t+s}$ независит от $t$, т.е. его свойства не зависят от времени. Т.е. если у нас есть окно производной длины $s$, то если мы его поставим в начале ряда, там будет совместное распределение $y$-ков, такое же, как и в конце этого ряда. Также, стоит упомянуть, что существуют и другие виды стационарности, например стационарность по среднему \cite{studopedia}.

Про стационарность можно отметить, на основе тех компонентов рядов, которые мы выделили:

\begin{itemize}
	\item ряды с трендами - нестационарны;
	\item ряды с сезонностью - нестационарны;
	\item наличие циклов в ряде не могут точно сказать, стационарен ряд или нет.
\end{itemize}

В общем случае, чтобы проверить - стационарен ли временной ряд, можно использовать статистические критерии. Существует множество данных критериев, далее будет рассмотрен один из них.

\subsection{Критерий проверки стационарности - KPSS}
\label{sec:stationary:kpss}

Критерий проверки стационарности ряда называемый KPSS проверяет гипотезу о том, что ряд стационарен, против альтернативы, что он нестационарен \cite{statsoft}.

Статистика для данного критерия выглядит следующим образом:

\begin{equation}
KPSS = \frac{1}{T^2}\sum_{i=1}^{T}{(\sum_{t=1}^i{y_t})^2/\lambda^2},
\end{equation}

при нулевой гипотезе, имеет табличное распределение (т.е. оно никак не выражается аналитически).

\subsection{Критерий Дики-Фуллера}
\label{sec:stationary:dicky-fuller}

Следующим критерием, который будет рассморен является критерий Дики-Фуллера. Он устроен ровно наоборот. Критерий Дики-Фуллера проверяет гипотезу о том, что ряд нестационарен, против гипотезы, что он стационарен. Можно отметить, что таким образом устроено большинство статистических критериев работающих со стационарностью.


